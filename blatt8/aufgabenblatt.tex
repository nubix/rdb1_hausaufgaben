\documentclass[11pt,a4paper,DIV=9]{scrartcl}
\usepackage{pgf}
\usepackage{tikz}
\usetikzlibrary{arrows,automata,positioning,shadows}
\usepackage{ngerman}
\usepackage[utf8]{inputenc}
\usepackage{amsmath,amssymb}
\usepackage{tikz-er2}
\usepackage{enumerate}
\usepackage{listings}
\usetikzlibrary{shapes,snakes}
% Schriftart ändern
\renewcommand{\rmdefault}{ppl}
%Möglichkeit zur Änderung von Überschriften
\usepackage{sectsty}
%Überschrift \section uandern
\definecolor{blue}{RGB}{76 , 92, 153}
\allsectionsfont{\color{blue}}
\paragraphfont{\color{blue}}

%Variable Blattnummer
\newcommand{\blatt}[1]{
  \newcommand{\blattnr}{#1}
}
%Aufgabe und Aufgabenteil definieren
\newcounter{temp}
\newcommand{\aufgabe}[1]{
  \setcounter{temp}{\value{subsection}}
  \setcounter{subsection}{#1}
  \addtocounter{subsection}{-1}
  \subsection{Aufgabe}
  \setcounter{subsection}{\value{temp}}
}
\newcommand{\teil}[2][]{
  \subsubsection*{#2) #1}
}
\renewcommand{\author}[1]{\renewcommand{\author}{#1}}
\renewcommand{\title}[1]{\renewcommand{\title}{#1}}
\newcommand{\makehomeworktitle}{
  \begin{minipage}[t]{6.5cm}
    \sf{\author}
  \end{minipage}
  \begin{minipage}[t]{6.5cm}
    \begin{flushright}
      \sf{\title\\\today}
    \end{flushright}
  \end{minipage}
  \\[0.2cm]
  \begin{center}
    \sf{
      \color{blue}{
        \LARGE{Aufgabenblatt \blattnr}
      }
    }
  \end{center}
  \vspace{0.1cm}
}

%%%%%%%%%%%%%%%%%%%%%%%%
%%% Statisch
\author{{[}4131658{]} Jan Germann \\{[}4054962{]} Christian Ratz\\Übungsgruppe 1}
\title{Relationale Datenbanken}

%%% Auf jedes Hausaufgabenblatt anpassen
\blatt{8}
%%%%%%%%%%%%%%%%%%%%%%%%
\setcounter{section}{\blattnr}

\definecolor{dkgreen}{rgb}{0,0.6,0}
\definecolor{gray}{rgb}{0.5,0.5,0.5}
\definecolor{mauve}{rgb}{0.58,0,0.82}

\lstset{ %
  basicstyle=\footnotesize\ttfamily,
  language=sql,                % the language of the code
  numbers=left,
  numberstyle=\tiny\color{gray},
  keywordstyle=\color{blue},          % keyword style
  commentstyle=\color{dkgreen},       % comment style
  stringstyle=\color{mauve}}

\begin{document}
\makehomeworktitle
\aufgabe{1}
  \begin{enumerate}[a.]
      \item Name of all persons who played a role in a »parody« of any movie.\hfill
\begin{lstlisting}
SELECT p.name
FROM Movie m, Person p, actor a, hasGenre hg
WHERE p.id = a.person
AND a.movie = m.id
AND m.id = hg.movie
AND hg.genre = 'parody';
\end{lstlisting}

      \item Names of all persons who were born in July.\hfill\\
\begin{lstlisting}
SELECT name 
FROM Person 
WHERE MONTH(birthday)='7';
\end{lstlisting}

      \item Titles of all movies that have been directed by more than one person.\hfill\\
\begin{lstlisting}
SELECT m.title 
FROM Movie m 
JOIN director d 
ON (m.id = d.movie) 
GROUP BY m.id; 
\end{lstlisting}


      \item Id, name and number of movies the person acted in for each person. The result should also include those persons who never played a role in a movie yet.\hfill
\begin{lstlisting}
SELECT p.id, p.name, COUNT(*) AS MovieCount 
FROM Person p 
JOIN actor a ON (p.id = a.person) 
JOIN Movies m ON (m.id = a.movie) 

UNION
SELECT p.id, p.name, '0' 
FROM Person p, actor a 
WHERE a.person <> p.id;
\end{lstlisting}


      \item Titles of all movies in which »Ted Codd« played a role but »Ray Boyce« did not.\hfill
\begin{lstlisting}
SELECT m.title 
FROM Movie m, Person p, actor a  
WHERE a.person = p.id 
AND a.movie = m.id 
AND p.name = 'Ted Codd' 
EXCEPT (SELECT m.title 
FROM Movie m, Person p, actor a 
WHERE a.person = p.id 
AND a.movie = m.id 
AND p.name = 'Ray Boyce');
\end{lstlisting}


      \item The names of all persons, who acted and directed in the same movie for at least two times.\hfill
\begin{lstlisting}
SELECT p.name 
FROM person p 
JOIN actor a ON (p.id = a.person) 
JOIN director d ON (a.person = d.person) AND (a.movie = d.movie) 
WHERE COUNT(p.id) > 1;
\end{lstlisting}
 

      \item Considering the average number of movies persons played roles in. Return the names of all persons who played roles in more movies than the average.\hfill
\begin{lstlisting}
WITH 
  Avrg AS (
    SELECT AVG(COUNT(actor.person)) as average FROM actor
  ) 
SELECT p.name, 
COUNT(a.person) AS played 
FROM Person p, actor a, Avrg av 
WHERE  played > av.average; 
\end{lstlisting}

      
    \end{enumerate}
\aufgabe{2}
\begin{enumerate}[a.]
\item What is the difference between the WHERE and the HAVING clause? \hfill\\\\
 Der erste Unterschied w\"are, dass die WHERE-Klausel vor der \textbf{GROUP-BY}-Klausel verwendet wird, die HAVING-Klausel immer danach benutzt wird. Bei der Verwendung von \textbf{HAVING} wird erst eine komplette Abfrage ausgef\"uhrt, dann gruppiert und am Ende der Bedingungen an die Abfrage angepasst. Bei \textbf{WHERE} werden erst die entsprechenden Zeilen herausgesucht und am Ende gruppiert zur\"uckgegeben. \\
\item Would SQL have the same expressiveness if the HAVING clause would not exist? \hfill\\
Nein. In der \textbf{WHERE}-Klausel werden Vorbedingungen definiert, durch die \textbf{GROUP-BY}-Klausel wird die Selektion gruppiert und optionale Nachbedingungen (z.B Funktionen) werden in der \textbf{HAVING}-Klausel beschrieben. Des Weiteren ist Aggregation in der \textbf{WHERE}-Klausel verboten. Beispielsweise will man alle Institute des Informatikzweigs der TU Braunschweig mit mehr als 10 Mitarbeitern die ein höheres Monatsgehalt als 5000 Euro bekommen. \\
\begin{lstlisting}
SELECT Institut, COUNT(*) AS 'Anzahl der Mitarbeiter'
FROM Mitarbeiter 
WHERE Monatsgehalt > 5000
GROUP BY Institut
HAVING COUNT(*) > 10;
\end{lstlisting}
W\"urde man die \textbf{HAVING}-Klausel weglassen und stattdessen die \textbf{WHERE}-Klausel folgenderma{\ss}en definieren: \\
\begin{lstlisting}[numbers=none]
WHERE Monatsgehalt > 5000 AND COUNT(*) > 10
\end{lstlisting}
Dann w\"urde man einen Fehler wegen Aggregation bekommen, da in der \textbf{WHERE}-Klausel Aggregation verboten ist. \\
\item What is the difference between correlated and uncorrelated subqueries? \hfill\\\\
Falls die WHERE-Klausel der inneren Abfrage (Subquery) ein Attribut aus einer Relation verwendet, das in der \"au{\ss}eren Abfrage deklariert wurde, sind beide Abfragen korreliert. D.h die innere Abfrage muss f\"ur jedes Tupel der \"au{\ss}eren Abfrage neu bewertet werden. Da dies ineffizient ist, sollte man Unterabfragen (Subqueries) vermeiden und falls m\"oglich, auf JOINS zur\"uckgreifen. \\
\item What is the difference between a set and a bag? How can you enforce the DBMS to return a set instead of a bag? \\\\
Bags sind multiple Mengen von Zeilen. Duplikate sind standardm\"a{\ss}ig erlaubt, k\"onnen jedoch beim Anfragen eliminiert werden. Jeder einzelne Wert ist eine Zeilenmenge. Bags werden oft auch Ergebnismenge genannt. Letztlich ist es dasselbe, bis auf die Tatsache, dass sich Ergebnismengen durch die ORDER BY - Operation zu einer Listen machen lassen.

 \end{enumerate}
\end{document}

