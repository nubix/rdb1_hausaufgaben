\documentclass[11pt,a4paper,DIV=9]{scrartcl}
\usepackage{pgf}
\usepackage{tikz}
\usetikzlibrary{arrows,automata,positioning,shadows}
\usepackage{ngerman}
\usepackage[utf8]{inputenc}
\usepackage{amsmath,amssymb}
\usepackage{tikz-er2}
\usepackage{enumerate}
\usepackage{listings}
\usetikzlibrary{shapes,snakes}
% Schriftart ändern
\renewcommand{\rmdefault}{ppl}
%Möglichkeit zur Änderung von Überschriften
\usepackage{sectsty}
%Überschrift \section uandern
\definecolor{blue}{RGB}{76 , 92, 153}
\allsectionsfont{\color{blue}}
\paragraphfont{\color{blue}}

%Variable Blattnummer
\newcommand{\blatt}[1]{
  \newcommand{\blattnr}{#1}
}
%Aufgabe und Aufgabenteil definieren
\newcounter{temp}
\newcommand{\aufgabe}[1]{
  \setcounter{temp}{\value{subsection}}
  \setcounter{subsection}{#1}
  \addtocounter{subsection}{-1}
  \subsection{Aufgabe}
  \setcounter{subsection}{\value{temp}}
}
\newcommand{\teil}[2][]{
  \subsubsection*{#2) #1}
}
\renewcommand{\author}[1]{\renewcommand{\author}{#1}}
\renewcommand{\title}[1]{\renewcommand{\title}{#1}}
\newcommand{\makehomeworktitle}{
  \begin{minipage}[t]{6.5cm}
    \sf{\author}
  \end{minipage}
  \begin{minipage}[t]{6.5cm}
    \begin{flushright}
      \sf{\title\\\today}
    \end{flushright}
  \end{minipage}
  \\[0.2cm]
  \begin{center}
    \sf{
      \color{blue}{
        \LARGE{Aufgabenblatt \blattnr}
      }
    }
  \end{center}
  \vspace{0.1cm}
}

%%%%%%%%%%%%%%%%%%%%%%%%
%%% Statisch
\author{{[}4131658{]} Jan Germann \\{[}4054962{]} Christian Ratz\\Übungsgruppe 1}
\title{Relationale Datenbanken}

%%% Auf jedes Hausaufgabenblatt anpassen
\blatt{9}
%%%%%%%%%%%%%%%%%%%%%%%%
\setcounter{section}{\blattnr}

\definecolor{dkgreen}{rgb}{0,0.6,0}
\definecolor{gray}{rgb}{0.5,0.5,0.5}
\definecolor{mauve}{rgb}{0.58,0,0.82}

\lstset{ 
  basicstyle=\footnotesize\ttfamily,
  language=sql,                % the language of the code
  numbers=left,
  numberstyle=\tiny\color{gray},
  keywordstyle=\color{blue},          % keyword style
  commentstyle=\color{dkgreen},       % comment style
  stringstyle=\color{mauve}}

\begin{document}
\makehomeworktitle
   \aufgabe{1}
   Based on the given conceptual schema, please provide SQL statements to create the according tables described in the schema.
   \begin{enumerate}
   \item \textbf{Employee}
\begin{lstlisting}
CREATE TABLE Employee (
  empNr INTEGER NOT NULL,
  name VARCHAR(400),
  phone nr INTEGER,
  project VARCHAR(100) REFERENCES Project(name),
  ticket VARCHAR(100) REFERENCES Ticket(nr),
  %%%%%%%%%%%
  PRIMARY KEY(empNr)
);
\end{lstlisting}
    \item \textbf{Project}
\begin{lstlisting}
CREATE TABLE Project (
  name VARCHAR(100) NOT NULL, 
  description VARCHAR(400), 
  ticket VARCHAR(100) REFERENCES Ticket(nr),
  PRIMARY KEY(name)
);
    \end{lstlisting}
    \item \textbf{Ticket}
\begin{lstlisting}
CREATE TABLE Ticket (
  nr VARCHAR(100) NOT NULL, 
  title VARCHAR(400), 
  description VARCHAR(400),
  importance VARCHAR(100),
  project VARCHAR(100) REFERENCES Project(name),
  PRIMARY KEY(id)
);
\end{lstlisting}
    \item \textbf{Department}
\begin{lstlisting}
CREATE TABLE Department (
  nr INTEGER NOT NULL, 
  house VARCHAR(100) NOT NULL,
  budget VARCHAR(400),
  manage
  works in
  PRIMARY KEY(nr, house)
);
\end{lstlisting}
    \item \textbf{worksOn}
\begin{lstlisting}
CREATE TABLE worksOn (
  employee INTEGER NOT NULL REFERENCES Employee(empNr),
  ticket INTEGER NOT NULL REFERENCES Ticket(nr),
  start DATE,
  end DATE,
  PRIMARY KEY(employee, ticket)
);
\end{lstlisting}	
    \end{enumerate}
   \aufgabe{2}
   Based on the given statements and data, explain the consequences of the following operations:
   \begin{enumerate}[a)]
   \item \begin{lstlisting} 
	INSERT INTO connection VALUES(2, 5, 'sequel') 
   \end{lstlisting}
   
   
   
   
   \item \begin{lstlisting} 
	DELETE FROM actor WHERE role = 'forest ranger'
      \end{lstlisting}
      
   
   

   \item \begin{lstlisting} 
	DELETE FROM movie WHERE title = 'Adventures with RDB'
   \end{lstlisting}
   
   
   

   \item \begin{lstlisting} 
	INSERT INTO actor VALUES(6, 85, 'important looking man')
   \end{lstlisting}
   
   
   

   \item \begin{lstlisting} 
	DROP TABLE person
  \end{lstlisting}
  Die gesamte Tabelle Person wird aus der Datenbank gel\"oscht. 
	Da diese Tabelle nur die Prim\"arschl\"usselspalte besitzt, 
	welche mit den Spalten person aus den Relationen director und actor in Beziehung stehen, 
	wird die Spalte person dieser beiden Tabellen, also alle Zeilen aus person aus director und actor gel\"oscht. 		Der Rest der Datenbank wird nicht in beeinflusst.
 

 \end{enumerate}   
\end{document}

