\documentclass[11pt,a4paper,DIV=9]{scrartcl}
\usepackage{pgf}
\usepackage{tikz}
\usetikzlibrary{arrows,automata,positioning,shadows}
\usepackage{ngerman}
\usepackage[utf8]{inputenc}
\usepackage{amsmath,amssymb}
\usepackage{tikz-er2}
\usepackage{enumerate}
\usepackage{listings}
\usetikzlibrary{shapes,snakes}
% Schriftart ändern
\renewcommand{\rmdefault}{ppl}
%Möglichkeit zur Änderung von Überschriften
\usepackage{sectsty}
%Überschrift \section uandern
\definecolor{blue}{RGB}{76 , 92, 153}
\allsectionsfont{\color{blue}}
\paragraphfont{\color{blue}}

%Variable Blattnummer
\newcommand{\blatt}[1]{
  \newcommand{\blattnr}{#1}
}
%Aufgabe und Aufgabenteil definieren
\newcounter{temp}
\newcommand{\aufgabe}[1]{
  \setcounter{temp}{\value{subsection}}
  \setcounter{subsection}{#1}
  \addtocounter{subsection}{-1}
  \subsection{Aufgabe}
  \setcounter{subsection}{\value{temp}}
}
\newcommand{\teil}[2][]{
  \subsubsection*{#2) #1}
}
\renewcommand{\author}[1]{\renewcommand{\author}{#1}}
\renewcommand{\title}[1]{\renewcommand{\title}{#1}}
\newcommand{\makehomeworktitle}{
  \begin{minipage}[t]{6.5cm}
    \sf{\author}
  \end{minipage}
  \begin{minipage}[t]{6.5cm}
    \begin{flushright}
      \sf{\title\\\today}
    \end{flushright}
  \end{minipage}
  \\[0.2cm]
  \begin{center}
    \sf{
      \color{blue}{
        \LARGE{Aufgabenblatt \blattnr}
      }
    }
  \end{center}
  \vspace{0.1cm}
}

%%%%%%%%%%%%%%%%%%%%%%%%
%%% Statisch
\author{{[}4131658{]} Jan Germann \\{[}4054962{]} Christian Ratz\\Übungsgruppe 1}
\title{Relationale Datenbanken}

%%% Auf jedes Hausaufgabenblatt anpassen
\blatt{9}
%%%%%%%%%%%%%%%%%%%%%%%%
\setcounter{section}{\blattnr}

\definecolor{dkgreen}{rgb}{0,0.6,0}
\definecolor{gray}{rgb}{0.5,0.5,0.5}
\definecolor{mauve}{rgb}{0.58,0,0.82}

\lstset{ 
  basicstyle=\footnotesize\ttfamily,
  language=sql,                % the language of the code
  numbers=left,
  numberstyle=\tiny\color{gray},
  keywordstyle=\color{blue},          % keyword style
  commentstyle=\color{dkgreen},       % comment style
  stringstyle=\color{mauve}}

\begin{document}
\makehomeworktitle
   \aufgabe{1}
   Based on the given conceptual schema, please provide SQL statements to create the according tables described in the schema.
\begin{lstlisting}
-- Employee gehort zu einer bestimmten Abteilung.
CREATE TABLE Employee (
  empNr        int(11) NOT NULL AUTO_INCREMENT, 
  name         varchar(255), 
  department   int(10) NOT NULL, 
  departmentnr int(10) NOT NULL, 
  PRIMARY KEY (empNr));

-- Telefonnummer kann mehrwertig sein. 
-- Also Tabelle fur Telefonnummer erstellen
CREATE TABLE PhoneNumbers (
  employeeempnr int(11) NOT NULL, 
  phoneNr       varchar(255));
  --PRIMARY KEY ( employeeempnr, phoneNr )
  
-- Die Abteilung hat einen Leiter
CREATE TABLE Department (
  house   --CHAR(1)-- NOT NULL, 
  nr      int(10) NOT NULL, 
  manager int(11) NOT NULL --REFERENCES employee,
  budget  --float(10), 
  PRIMARY KEY (house, nr));

-- Das Projekt wird uber eine Relationentabelle mit den
-- Mitarbeitern verknuepft.
CREATE TABLE Project (
  name        varchar(255) NOT NULL, 
  description varchar(255), 
  PRIMARY KEY (name));

-- Relationentabelle um Mitarbeiter mit einem Projekt zu verknuepfen
CREATE TABLE Employee_Project (
  employeeempnr int(11) NOT NULL --REFERENCES employee ON DELETE CASCADE,
  projectname   varchar(255) NOT NULL --REFERENCES project ON DELETE CASCADE,
  PRIMARY KEY (employeeempnr, projectname));

-- Die offenen Tickets werden hier gespeichert.
CREATE TABLE Ticket (
  nr          int(10) NOT NULL AUTO_INCREMENT,  ---abhaengig vom Projekt
  --project varchar(255) NOT NULL REFERENCES project on DELETE CASCADE,
  title       varchar(255), 
  description varchar(255), 
  importance  --varchar(255) CHECK(importance IN('minor', 'medium', 'major')),
  PRIMARY KEY (nr,--project));

-- Relationentabelle um Tickets einem Projekt zuzuordnen
CREATE TABLE Project_Ticket (
  projectname varchar(255) NOT NULL, 
  ticketnr    int(10) NOT NULL, 
  PRIMARY KEY (projectname, 
  ticketnr));

-- Relationentabelle um die Arbeit eines Mitarbeiters
-- an einem Ticket zu dokumentieren.
CREATE TABLE Employee_Ticket (
  employeeempnr int(11) NOT NULL, 
  ticketnr      int(10) NOT NULL, 
  start         timestamp NOT NULL, 
  end           timestamp NULL, 
  --project varchar(200) NOT NULL REFERENCES Project
  PRIMARY KEY (employeeempnr, ticketnr,--project));

-- Fremdschluesselabhaengigkeiten erstellen
ALTER TABLE PhoneNumbers
  ADD INDEX FKPhoneNumbe658372 (employeeempnr),
  ADD CONSTRAINT FKPhoneNumbe658372
  FOREIGN KEY (employeeempnr) REFERENCES Employee (empNr);
ALTER TABLE Employee_Project
  ADD INDEX FKEmployee_P990046 (employeeempnr),
  ADD CONSTRAINT FKEmployee_P990046
  FOREIGN KEY (employeeempnr) REFERENCES Employee (empNr);
ALTER TABLE Employee_Project
  ADD INDEX FKEmployee_P943600 (projectname),
  ADD CONSTRAINT FKEmployee_P943600
  FOREIGN KEY (projectname) REFERENCES Project (name);
ALTER TABLE Department
  ADD INDEX FKDepartment689529 (manager),
  ADD CONSTRAINT FKDepartment689529
  FOREIGN KEY (manager) REFERENCES Employee (empNr);
ALTER TABLE Employee
  ADD INDEX FKEmployee366382 (department, departmentnr),
  ADD CONSTRAINT FKEmployee366382
  FOREIGN KEY (department, departmentnr) REFERENCES Department (house, nr);
ALTER TABLE Project_Ticket
  ADD INDEX FKProject_Ti16548 (projectname),
  ADD CONSTRAINT FKProject_Ti16548
FOREIGN KEY (projectname) REFERENCES Project (name);
ALTER TABLE Project_Ticket
  ADD INDEX FKProject_Ti360860 (ticketnr),
  ADD CONSTRAINT FKProject_Ti360860
  FOREIGN KEY (ticketnr) REFERENCES Ticket (nr);
ALTER TABLE Employee_Ticket
  ADD INDEX FKEmployee_T249102 (employeeempnr),
  ADD CONSTRAINT FKEmployee_T249102
  FOREIGN KEY (employeeempnr) REFERENCES Employee (empNr);
ALTER TABLE Employee_Ticket
  ADD INDEX FKEmployee_T659766 (ticketnr),
  ADD CONSTRAINT FKEmployee_T659766
  FOREIGN KEY (ticketnr) REFERENCES Ticket (nr);
\end{lstlisting}

\aufgabe{2}
   Based on the given statements and data, explain the consequences of the following operations:
   \begin{enumerate}[a)]
   \item \begin{lstlisting} 
	INSERT INTO connection VALUES(2, 5, 'sequel') 
   \end{lstlisting} 
   Der Prim\"arschl\"ussel der aus \texttt{from\_movie, to\_movie} besteht, ist bereits vorhanden und mit dem \texttt{connection\_type} 'parody' belegt. Daher wird bei diesem INSERT-Statement nichts ver\"andert.
   
   
   \item \begin{lstlisting} 
	DELETE FROM actor WHERE role = 'forest ranger'
      \end{lstlisting}
      Alle Zeilen in der Relation \textbf{actor} in denen ein Schauspieler die \texttt{role} 'forest ranker' besitzt, werden gel\"oscht. Alle anderen Relationen werden nicht ver\"andert.
   
   

   \item \begin{lstlisting} 
	DELETE FROM movie WHERE title = 'Adventures with RDB'
   \end{lstlisting}
   Der Eintrag \"uber den Film 'Adventures with RDB' wird nicht entfernt, da die Standardeinstellung (siehe Vorlesungsfolie 20) bei L\"oschung keine Aktion ausgef\"uhrt werden soll, sowie bei \"Anderung eines Eintrag ebenfalls nichts ausgef\"uhrt werden soll.
   
   

   \item \begin{lstlisting} 
	INSERT INTO actor VALUES(6, 85, 'important looking man')
   \end{lstlisting}
   Das Einf\"ugen der Werte (6, 85, 'important looking man') in die Relation \textbf{actor} funktioniert nicht, da die \texttt{id} '6' in der Relation \textbf{person} und in der Relation \textbf{movie} die \texttt{id} '85' nicht existiert. Daher bleibt die Datenbank unver\"andert.
   
   

   \item \begin{lstlisting} 
	DROP TABLE person
  \end{lstlisting}
  \textcolor{red}{Es passiert nichts, alle Relationen bleiben unver\"andert, da noch andere Tabellen auf Person referenzieren.}
 

 \end{enumerate}   
\end{document}

