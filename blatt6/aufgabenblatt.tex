\documentclass[11pt,a4paper,DIV=9]{scrartcl}
\usepackage{pgf}
\usepackage{tikz}
\usetikzlibrary{arrows,automata,positioning,shadows}
\usepackage{ngerman}
\usepackage[utf8]{inputenc}
\usepackage{amsmath,amssymb}
\usepackage{tikz-er2}
\usetikzlibrary{shapes,snakes}
% Schriftart ändern
\renewcommand{\rmdefault}{ppl}
%Möglichkeit zur Änderung von Überschriften
\usepackage{sectsty}
%Überschrift \section uandern
\definecolor{blue}{RGB}{76 , 92, 153}
\allsectionsfont{\color{blue}}
\paragraphfont{\color{blue}}

%Variable Blattnummer
\newcommand{\blatt}[1]{
  \newcommand{\blattnr}{#1}
}
%Aufgabe und Aufgabenteil definieren
\newcounter{temp}
\newcommand{\aufgabe}[1]{
  \setcounter{temp}{\value{subsection}}
  \setcounter{subsection}{#1}
  \addtocounter{subsection}{-1}
  \subsection{Aufgabe}
  \setcounter{subsection}{\value{temp}}
}
\newcommand{\teil}[2][]{
  \subsubsection*{#2) #1}
}
\renewcommand{\author}[1]{\renewcommand{\author}{#1}}
\renewcommand{\title}[1]{\renewcommand{\title}{#1}}
\newcommand{\makehomeworktitle}{
  \begin{minipage}[t]{6.5cm}
    \sf{\author}
  \end{minipage}
  \begin{minipage}[t]{6.5cm}
    \begin{flushright}
      \sf{\title\\\today}
    \end{flushright}
  \end{minipage}
  \\[0.2cm]
  \begin{center}
    \sf{
      \color{blue}{
        \LARGE{Aufgabenblatt \blattnr}
      }
    }
  \end{center}
  \vspace{0.1cm}
}

%%%%%%%%%%%%%%%%%%%%%%%%
%%% Statisch
\author{{[}4131658{]} Jan Germann \\{[}4054962{]} Christian Ratz\\Übungsgruppe 1}
\title{Relationale Datenbanken}

%%% Auf jedes Hausaufgabenblatt anpassen
\blatt{6}
%%%%%%%%%%%%%%%%%%%%%%%%
\setcounter{section}{\blattnr}
\begin{document}
\makehomeworktitle
\aufgabe{1}
 a. The names of all female persons, who were born on ``01.01.1950``.
 \\\\ b. The names of all persons, who played a role in the movie ``The legacy of Codd``.
 \begin{displaymath} \pi _{Person.id, Person.name} ( \sigma _{Movie.title=`{The\, legacy\, of\, Codd`}} Movie\Join_{Movie.id = actor.movie} \sigma _{actor.role = \neg null} actor)  \end{displaymath}
 \\\\ c. The names of all persons, who at least once acted and directed in the same movie.
 \begin{displaymath} \pi _{Person.name} (\sigma _{actor.person = director.person \,\wedge\, actor.movie = director.movie} actor \times Director \times Person) \end{displaymath}
 SELECT person.names FROM Person WHERE actor.person = director.person AND WHERE actor.movie = director.movie
 \\\\ d. The number of parodies to the movie ``Adventures with relational databases``.
 \\\\ e. Genres that are not assigned to any movie at all.
 \begin{displaymath} \pi _{Genre.name} (\sigma _{hasGenre.genre=Genre.name \,\wedge\, movie = NULL} hasGenre) \end{displaymath}
 \\\\ f. The titles of movies that are a ``sequel`` of a ``parody``.
 \\\\ g. The person(s) who played the role ``relational algebra hacker`` most.
\aufgabe{2}
 a. NULL-Values werden ebenfalls mitkopiert, wenn Aggregation angewandt wird.
 \\\\ b. Duplikate werden bei der Aggregation nicht eliminiert.
 \\\\ c. Die Operanden m\"ussen kompatibel f\"ur die Vereinigung, Schnitt, Differenz sein. D.h Sie m\"ussen aus den selben Attributen zusammengesetzt sein.
\\\\ d. \begin{displaymath} R \Join _{\theta} S =  \sigma _{\theta} (R \times S) \end{displaymath}
\aufgabe{3}
\end{document}