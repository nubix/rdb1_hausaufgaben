\documentclass[11pt,a4paper,DIV=9]{scrartcl}
\usepackage{pgf}
\usepackage{tikz}
\usetikzlibrary{arrows,automata,positioning,shadows}
\usepackage{ngerman}
\usepackage[utf8]{inputenc}
\usepackage{amsmath,amssymb}
\usepackage{tikz-er2}
\usepackage{enumerate}
\usetikzlibrary{shapes,snakes}
% Schriftart ändern
\renewcommand{\rmdefault}{ppl}
%Möglichkeit zur Änderung von Überschriften
\usepackage{sectsty}
%Überschrift \section uandern
\definecolor{blue}{RGB}{76 , 92, 153}
\allsectionsfont{\color{blue}}
\paragraphfont{\color{blue}}

%Variable Blattnummer
\newcommand{\blatt}[1]{
  \newcommand{\blattnr}{#1}
}
%Aufgabe und Aufgabenteil definieren
\newcounter{temp}
\newcommand{\aufgabe}[1]{
  \setcounter{temp}{\value{subsection}}
  \setcounter{subsection}{#1}
  \addtocounter{subsection}{-1}
  \subsection{Aufgabe}
  \setcounter{subsection}{\value{temp}}
}
\newcommand{\teil}[2][]{
  \subsubsection*{#2) #1}
}
\renewcommand{\author}[1]{\renewcommand{\author}{#1}}
\renewcommand{\title}[1]{\renewcommand{\title}{#1}}
\newcommand{\makehomeworktitle}{
  \begin{minipage}[t]{6.5cm}
    \sf{\author}
  \end{minipage}
  \begin{minipage}[t]{6.5cm}
    \begin{flushright}
      \sf{\title\\\today}
    \end{flushright}
  \end{minipage}
  \\[0.2cm]
  \begin{center}
    \sf{
      \color{blue}{
        \LARGE{Aufgabenblatt \blattnr}
      }
    }
  \end{center}
  \vspace{0.1cm}
}

%%%%%%%%%%%%%%%%%%%%%%%%
%%% Statisch
\author{{[}4131658{]} Jan Germann \\{[}4054962{]} Christian Ratz\\Übungsgruppe 1}
\title{Relationale Datenbanken}

%%% Auf jedes Hausaufgabenblatt anpassen
\blatt{6}
%%%%%%%%%%%%%%%%%%%%%%%%
\setcounter{section}{\blattnr}

\begin{document}
\makehomeworktitle
\aufgabe{1}
  \begin{enumerate}[a)]
    \item The names of all female persons, who were born on ``01.01.1950``.\hfill\\
      $
      \pi_{Person.name}\\
      \sigma_{Person.birthday = '01.01.1950'\,\wedge\,Person.gender = 'f'}\\
      Person
      $
    \item The names of all persons, who played a role in the movie ``The legacy of Codd``. \hfill\\
      $
      \pi _{Person.id, Person.name}\\
      \sigma _{Movie.title=`{The\, legacy\, of\, Codd`}}\\
      (Movie\Join_{Movie.id = actor.movie} \sigma _{actor.role = \neg null} actor)
      $
    \item The names of all persons, who at least once acted and directed in the same movie.\hfill\\
      $
      \pi _{Person.name}\\
      \sigma _{actor.person = director.person \,\wedge\, actor.movie = director.movie}\\
      (actor \times Director \times Person)
      $
    \item The number of parodies to the movie ``Adventures with relational databases``.
      $
      \pi_{\mathfrak{F}_{count(Movie.title)}} \\
      \sigma_{Movie.title = '\textrm{Adventures with relational databases}' \,\wedge\, connection.type = 'parody'}\\
      (Movie\Join_{connection.to = Movie.id}connection)
      $

    \item Genres that are not assigned to any movie at all. \hfill \\
      $
      \pi _{Genre.name} \\
      (\pi _{Genre.name}(Genre \Join _{Genre.name = hasGenre.genre} Genre )\\
      \backslash
        \\
        \pi _{Genre.name} \sigma _{Genre.name = hasGenre.genre} (Genre \Join _{Movie.id = hasGenre.movie} hasGenre))
      $
    \item The titles of movies that are a ``sequel`` of a ``parody``. \\
      $\pi_{Movie.title} \\
      \sigma_{Movie.id = sequel.id} \\
      (\rho_{sequel(id)}\\
      \pi_{connection.from}\sigma_{connection.type='sequel'}\\
      (connection \Join_{connection.to = parody.from} \\ 
        (\rho_{parody(from,to,title)}\sigma_{type='pardoy'}connection)\\
      )\\
      )$
    \item The person(s) who played the role ``relational algebra hacker`` most.
      $
      \rho_{\mathfrak{F}max()}
      $
  \end{enumerate}
\aufgabe{2}
 \begin{enumerate}[a)]
   \item NULL-Values werden bei der Aggregation ignoriert.
   \item Duplikate werden bei der Aggregation ignoriert.
   \item Damit Operationen wie Vereinigung, Schnitt oder Differenz auf zwei Relationen angewandt werden k\"onnen m\"ussen die Relationen aus denselben Attributen zusammengesetzt sein.
   \item \begin{displaymath} R \Join _{\theta} S =  \sigma _{\theta} (R \times S) \end{displaymath}
 \end{enumerate}
\aufgabe{3}

\end{document}


