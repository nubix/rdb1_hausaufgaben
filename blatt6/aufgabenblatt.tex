\documentclass[11pt,a4paper,DIV=9]{scrartcl}
\usepackage{pgf}
\usepackage{tikz}
\usetikzlibrary{arrows,automata,positioning,shadows}
\usepackage{ngerman}
\usepackage[utf8]{inputenc}
\usepackage{amsmath,amssymb}
\usepackage{tikz-er2}
\usepackage{enumerate}
\usetikzlibrary{shapes,snakes}
% Schriftart ändern
\renewcommand{\rmdefault}{ppl}
%Möglichkeit zur Änderung von Überschriften
\usepackage{sectsty}
%Überschrift \section uandern
\definecolor{blue}{RGB}{76 , 92, 153}
\allsectionsfont{\color{blue}}
\paragraphfont{\color{blue}}

%Variable Blattnummer
\newcommand{\blatt}[1]{
  \newcommand{\blattnr}{#1}
}
%Aufgabe und Aufgabenteil definieren
\newcounter{temp}
\newcommand{\aufgabe}[1]{
  \setcounter{temp}{\value{subsection}}
  \setcounter{subsection}{#1}
  \addtocounter{subsection}{-1}
  \subsection{Aufgabe}
  \setcounter{subsection}{\value{temp}}
}
\newcommand{\teil}[2][]{
  \subsubsection*{#2) #1}
}
\renewcommand{\author}[1]{\renewcommand{\author}{#1}}
\renewcommand{\title}[1]{\renewcommand{\title}{#1}}
\newcommand{\makehomeworktitle}{
  \begin{minipage}[t]{6.5cm}
    \sf{\author}
  \end{minipage}
  \begin{minipage}[t]{6.5cm}
    \begin{flushright}
      \sf{\title\\\today}
    \end{flushright}
  \end{minipage}
  \\[0.2cm]
  \begin{center}
    \sf{
      \color{blue}{
        \LARGE{Aufgabenblatt \blattnr}
      }
    }
  \end{center}
  \vspace{0.1cm}
}

%%%%%%%%%%%%%%%%%%%%%%%%
%%% Statisch
\author{{[}4131658{]} Jan Germann \\{[}4054962{]} Christian Ratz\\Übungsgruppe 1}
\title{Relationale Datenbanken}

%%% Auf jedes Hausaufgabenblatt anpassen
\blatt{6}
%%%%%%%%%%%%%%%%%%%%%%%%
\setcounter{section}{\blattnr}

\begin{document}
\makehomeworktitle
\aufgabe{1}
  \begin{enumerate}[a)]
    \item The names of all female persons, who were born on ``01.01.1950``.\hfill\\
      $
      \pi_{Person.name}\\
      \sigma_{Person.birthday = '01.01.1950'\,\wedge\,Person.gender = 'f'}\\
      Person
      $
    \item The names of all persons, who played a role in the movie ``The legacy of Codd``. \hfill\\
      $
      \pi _{Person.id, Person.name}\\
      \sigma _{Movie.title=`{The\, legacy\, of\, Codd`}}\\
      (Movie\Join_{Movie.id = actor.movie} \sigma _{actor.role = \neg null\textcolor{red}{ <-geht nicht }} actor)
      $
      \textcolor{red}{Person.id und Person.name muss aus Person geholt werden, nicht angeben.}
    \item The names of all persons, who at least once acted and directed in the same movie.\hfill\\
      $
      \pi _{Person.name}\\
      \sigma _{actor.person = director.person \,\wedge\, actor.movie = director.movie \, \textcolor{red}{\wedge\, Person.id=actor.person}}\\
      (actor \times Director \times Person)
      $
    \item The number of parodies to the movie ``Adventures with relational databases``.
      $
      \pi_{\mathfrak{F}_{count(\textcolor{red}{from})}} \\
      \sigma_{Movie.title = '\textrm{Adventures with relational databases}' \,\wedge\, connection.type = 'parody'}\\
      (Movie\Join_{connection.to = Movie.id}connection)
      $

    \item Genres that are not assigned to any movie at all. \hfill \\
      $
      (\pi _{Genre.name} \textcolor{red}{Genre}) \\
      \backslash
        \\
        \textcolor{red}{(\pi _{hasGenre.genre}hasGenre)}
      $
    \item The titles of movies that are a ``sequel`` of a ``parody``. \\
      $\pi_{Movie.title} \\
      \sigma_{Movie.id = sequel.id} \\
      (\rho_{sequel(id)}\\
      \pi_{connection.from}\sigma_{connection.type='sequel'}\\
      (connection \Join_{connection.to = parody.from} \\ 
        (\rho_{parody(from,to,title)}\sigma_{type='pardoy'}connection)\\
      ))$
    \item The person(s) who played the role ``relational algebra hacker`` most.\hfill\\
      $
      Person \Join_{Person.id = count.personId}(\pi_{count.personId}\\
      \sigma_{count.num=max(count.num)}\\
      (
        \rho_{count(personId,num)} (\\
          _{Person.id}\mathfrak{F}_{count(Person.id)}(\\
          Person\Join_{Person.id = actor.person \,\wedge\, role = '\textrm{relational algebra hacker}'}actor\\
          )\\
        )\\
      ))
      $
  \end{enumerate}
\aufgabe{2}
 \begin{enumerate}[a)]
   \item NULL-Values von Attributen werden bei der Aggregation ignoriert. D.h die Werte der Attribute werden wie normale Werte behandelt.
   \item Doppelte Werte bzw. Duplikate gehen bei der Aggregation ebenfalls doppelt ein.
   \item Vereinigung: Beide Relationen m\"ussen dieselbe Anzahl an Attributen besitzen dessen Namen \"ubereinstimmen. 
   \\ Schnitt: M\"ussen dieselbe Anzahl an Argumenten in den einzelnen Tupeln besitzen, Namen m\"ussen \"ubereinstimmen
   \\ Differenz: \textcolor{red}{Es gilt f\"ur alle Differenz dasselbe, wie bei Schnitt oder Vereinigung}

   \item \begin{displaymath} R \Join _{\theta} S =  \sigma _{\theta} (R \times S) \end{displaymath}
 \end{enumerate}
\aufgabe{3}
$ \pi _{s}\, (A\, \cap \, B) = (\pi _{s} \, A)\, \cap \,(\pi _{s}\, B)$
\\\\ \textbf{A} = \textbf{Kurse} | \textbf{B} = \textbf{Songs} | \texttt{s} = \texttt{Titel} 
\\\\Wir haben beispielsweise eine Relation \textbf{Kurse} mit den Attributen \texttt{Kursnummer} und \texttt{Titel} (exemplarisch f\"ur A). Ebenso haben wir eine Relation \textbf{Songs} mit den Attributen \texttt{Songnummer} und ebenfalls einem \texttt{Titel} (exemplarisch f\"ur B). Der \texttt{Titel} steht in unserem Fall f\"ur das Attribut s. Links vom Gleichheitszeichen wird also erst die Schnittmenge aus Tupeln zwischen \textbf{Kurse} und \textbf{Songs} erstellt, von der Ergebnisrelation werden dann die \texttt{Titel} projiziert. Rechts vom Gleichheitszeichen werden jeweils erst die \texttt{Titel} der \textbf{Kurse} sowie die \texttt{Titel} der \textbf{Songs} projiziert. Aus den beiden Ergebnisrelationen wird dann die Schnittmenge an Tupeln erstellt. Es kann also passieren, wenn auch sehr unwahrscheinlich, dass ein \texttt{Titel} eines \textbf{Songs} genauso lautet wie der \texttt{Titel} eines \textbf{Kurses}. Es ist also nicht dieselbe Schnittmenge an Tupeln wie auf der linken Seite vom Gleichheitszeichen. Auf der linken Seite vom Gleichheitszeichen muss, damit man gleiche Tupel erh\"alt nicht nur der \texttt{Titel} gleich sein sondern auch die \texttt{Nummer} der Song- oder Kurstitel. Da dies jedoch auf jedenfall nicht so h\"aufig vorkommen wird, wie zwei identische \texttt{Titel}, erh\"alt man auf der linken Seite vom Gleichheitszeichen eine andere Schnittmenge an Tupeln.
\textcolor{red}{Sch\"oner w\"aren anschauliche Tabellen.}
\end{document}

