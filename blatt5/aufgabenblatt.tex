\documentclass[11pt,a4paper,DIV=9]{scrartcl}
\usepackage{pgf}
\usepackage{tikz}
\usetikzlibrary{arrows,automata,positioning,shadows}
\usepackage{ngerman}
\usepackage[utf8]{inputenc}
\usepackage{amsmath,amssymb}
\usepackage{tikz-er2}
\usepackage{pdflscape}
\usetikzlibrary{shapes,snakes}
% Schriftart ändern
\renewcommand{\rmdefault}{ppl}
%Möglichkeit zur Änderung von Überschriften
\usepackage{sectsty}
%Überschrift \section uandern
\definecolor{blue}{RGB}{76 , 92, 153}
\allsectionsfont{\color{blue}}
\paragraphfont{\color{blue}}

%Variable Blattnummer
\newcommand{\blatt}[1]{
  \newcommand{\blattnr}{#1}
}
%Aufgabe und Aufgabenteil definieren
\newcounter{temp}
\newcommand{\aufgabe}[1]{
  \setcounter{temp}{\value{subsection}}
  \setcounter{subsection}{#1}
  \addtocounter{subsection}{-1}
  \subsection{Aufgabe}
  \setcounter{subsection}{\value{temp}}
}
\newcommand{\teil}[2][]{
  \subsubsection*{#2) #1}
}
\renewcommand{\author}[1]{\renewcommand{\author}{#1}}
\renewcommand{\title}[1]{\renewcommand{\title}{#1}}
\newcommand{\makehomeworktitle}{
  \begin{minipage}[t]{6.5cm}
    \sf{\author}
  \end{minipage}
  \begin{minipage}[t]{6.5cm}
    \begin{flushright}
      \sf{\title\\\today}
    \end{flushright}
  \end{minipage}
  \\[0.2cm]
  \begin{center}
    \sf{
      \color{blue}{
        \LARGE{Aufgabenblatt \blattnr}
      }
    }
  \end{center}
  \vspace{0.1cm}
}

%%%%%%%%%%%%%%%%%%%%%%%%
%%% Statisch
\author{{[}4131658{]} Jan Germann \\{[}4054962{]} Christian Ratz\\Übungsgruppe 1}
\title{Relationale Datenbanken}

%%% Auf jedes Hausaufgabenblatt anpassen
\blatt{5}
%%%%%%%%%%%%%%%%%%%%%%%%
\setcounter{section}{\blattnr}
\begin{document}
\makehomeworktitle

\aufgabe{1}
\teil[Name and explain the three structural constraints that need to be maintained within a database. Use your own words.]{a}
 \begin{enumerate}
    \item \texttt{Unique Key constraint} -- Eine Relation ist immer eine Menge von Tupeln. Alle Tupel die in dieser Menge vorkommen m\"ussen unterscheidbar sein. Die Unterscheidbarkeit wird durch z.B durch einen Prim\"arschl\"ussel gew\"ahrleistet. (Einheitsschl\"usselbeschr\"ankung)
    \item \texttt{Entity integrity constraint} -- Diese Bedingung beschreibt, dass keine Prim\"arschl\"usselspalte einer Relation den NULL-Wert annehmen kann. Des Weiteren darf muss die Eindeutigkeit jeden Prim\"arschl\"ussels gew\"ahrleistet werden. Beispiel: Relation Stadt(\underline{Name} NOT NULL, \underline{Bundesland} NOT NULL, Postleitzahl)
    \item \texttt{Referential integrity constraint} -- Zu jedem Wert einer Fremdschl\"usselspalte einer Relation, der nicht den NULL-Wert besitzt, genau dieser Wert f\"ur die referenzierte Prim\"arschl\"usselspalte der referenzierten Relation existiert. Beziehungen existieren nur, wenn die entsprechenden Entit\"aten existieren. (Referentielle Integrit\"atsbeschr\"ankung)
    \\\\ Alle Beschr\"ankungen sind wichtig um \underline{korrekt} neue Tupel einzuf\"ugen, zu l\"oschen oder zu \"andern.
  \end{enumerate}
\teil[Briefly explain what is meant by the first normal form.]{b}
Eine Relation befindet sich in der ersten Normalform, wenn der Wert eine jeden Attributs atomisch ist, zusammengesetzte oder mengenwertige Werte sind unzul\"assig. Der \texttt{Student} darf nicht als Attribut verwendet werden, sondern muss - falls die Relation in die erste Normalform gebracht wird - in \texttt{Matrikelnummer}, \texttt{Vorname}, \texttt{Name}, etc.. aufgeteilt werden.
  
 \aufgabe{2}
\teil[Translate the following conceptual schemata into a relational schemata. Please write down all additional constraints that are not represented in your relational model.]{1}
  \begin{tikzpicture}[scale=1,every edge/.style={draw=blue, very thick}]
    \node[entity] (A) {A};
      \node[attribute] () [above left=of A] {\underline{a1}}edge node[right]{} (A);
      \node[attribute] () [above=of A] {a2}edge node[right]{} (A);

    \node[relationship] (r1) [right=of A] {R} edge node[above]{(0,*)} (A);
    
   \node[entity] (C) [above= of r1] {C}edge node[right]{(0,*)} (r1);
    \node[attribute] () [above left=of C] {\underline{c1}}edge node[right]{} (C);
    \node[attribute] () [above right=of C] {c2}edge node[right]{} (C);

    \node[entity] (B) [right=of r1] {city}edge node[above]{(0,*)} (r1);
     \node[attribute] () [above right=of B] {\underline{b1}}edge node[right]{} (B);
     \node[attribute] () [above=of B] {\underline{b2}}edge node[right]{} (B);
     
  \end{tikzpicture}
  \\\\
a) 	A (\underline{a1}: INTEGER(100), a2: INTEGER(100)); \\
	B (\underline{b1}: INTEGER(100), \underline{b2}: INTEGER(100)); \\
	C (\underline{c1}: INTEGER(100), c2()); \\  
	R (\underline{a1}: INTEGER(100), \underline{b1}: INTEGER(100), \underline{b2}: INTEGER(100), \underline{c1}: INTEGER(100)); \\\\\\

 \begin{tikzpicture}[scale=1, every edge/.style={draw=blue, very thick}]
   \node[entity] (A) {A};
    \node[attribute] () [above left= of A] {\underline{a1}}edge node[right]{} (A);
    \node[attribute] () [above= of A] {a2}edge node[right]{} (A);
    
    \node[relationship] (r2) [right= of A] {R}edge node[above]{(1,1)} (A);
    
    \node[attribute] () [above= of r2] {r1}edge node[above]{} (r2);
  
  \node[entity] (B) [right= of r2] {B}edge node[above]{(1,*)} (r2);
    \node[attribute] () [above= of B] {\underline{b1}}edge node[right]{} (B);
    \node[attribute] () [above right= of B] {\underline{b2}}edge node[right]{} (B);
  \end{tikzpicture}
   \\\\
b) 	A (\underline{a1}: INTEGER(100), a2: INTEGER(100)); \\
	B (\underline{b1}: INTEGER(100), \underline{b2}: INTEGER(100)); \\
	R (\underline{a1}: INTEGER(100), \underline{b1}: INTEGER(100), \underline{b2}: INTEGER(100), r1: INTEGER(100)); \\\\\\
 \begin{tikzpicture}[scale=1, every edge/.style={draw=blue, very thick}]
   \node[entity] (A) {A};
    \node[attribute] () [above left= of A] {\underline{a1}}edge node[right]{} (A);
    \node[attribute] () [above= of A] {a2}edge node[right]{} (A);
    
    \node[relationship] (r2) [right= of A] {R}edge node[above]{(0,*)} (A);
    
  \node[entity] (B) [right= of r2] {B}edge node[above]{} (r2);
    \node[attribute] () [above= of B] {\underline{b1}}edge node[right]{} (B);
    \node[attribute] () [above right= of B] {\underline{b2}}edge node[right]{} (B);
  \end{tikzpicture}
   \\\\
c) 	A (\underline{a1}: INTEGER(100), a2: INTEGER(100)); \\
	B (\underline{b1}: INTEGER(100), \underline{b2}: INTEGER(100)); \\
	R (\underline{a1}: INTEGER(100), \underline{b1}: INTEGER(100), \underline{b2}: INTEGER(100)); \\
\end{document}