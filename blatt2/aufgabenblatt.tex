\documentclass[12pt,a4paper,DIV=9]{scrartcl}
\usepackage{pgf}
\usepackage{tikz}
\usetikzlibrary{arrows,automata}
\usepackage{ngerman}
\usepackage[utf8]{inputenc}
\usepackage{amsmath,amssymb}
% Schriftart ändern
\renewcommand{\rmdefault}{ppl}
%Möglichkeit zur Änderung von Überschriften
\usepackage{sectsty}
%Überschrift \section uandern
\definecolor{blue}{RGB}{76 , 92, 153}
\allsectionsfont{\color{blue}}
\paragraphfont{\color{blue}}

%Variable Blattnummer
\newcommand{\blatt}[1]{
  \newcommand{\blattnr}{#1}
}
%Aufgabe und Aufgabenteil definieren
\newcounter{temp}
\newcommand{\aufgabe}[1]{
  \setcounter{temp}{\value{subsection}}
  \setcounter{subsection}{#1}
  \addtocounter{subsection}{-1}
  \subsection{Aufgabe}
  \setcounter{subsection}{\value{temp}}
}
\newcommand{\teil}[2][]{
  \subsubsection*{#2) #1}
}
\renewcommand{\author}[1]{\renewcommand{\author}{#1}}
\renewcommand{\title}[1]{\renewcommand{\title}{#1}}
\newcommand{\makehomeworktitle}{
  \begin{minipage}{6.5cm}
    \sf{\author}
  \end{minipage}
  \begin{minipage}{6.5cm}
    \begin{flushright}
      \sf{\title\\\today}
    \end{flushright}
  \end{minipage}
  \\[0.2cm]
  \begin{center}
    \sf{
      \color{blue}{
        \LARGE{Aufgabenblatt \blattnr}
      }
    }
  \end{center}
  \vspace{0.1cm}
}

%%%%%%%%%%%%%%%%%%%%%%%%
%%% Statisch
\author{{[}4131658{]} Jan Germann \\{[}1234567{]} Christian Ratz}
\title{Relationale Datenbanken 1}

%%% Auf jedes Hausaufgabenblatt anpassen
\blatt{2}
%%%%%%%%%%%%%%%%%%%%%%%%
\setcounter{section}{\blattnr}
\begin{document}
\makehomeworktitle

\aufgabe{1}
\teil{a}
  Die ANSI-SPARC Architektur separiert die Benutzeranwendungen und Views von der physikalischen Datenbank. Sie besteht aus den drei Ebenen 
  \begin{enumerate}
    \item \bf{Darstellungsebene} \\
      Diese Schicht ist extern und nicht direkt ein Teil des DBMS
    \item \bf{logische Ebene} \\
      Hat einen Menschen,  alter, ganz Zahl 
    \item \bf{physikalische Ebene}
      Bestimmt wie, wo daten gespeichert werden und was dort gespeichert wird. Zum Beispiel bestimmt es ob die daten auf platte eins oder platte x liegen. Interne representation von daten, z.B. Integer wird in n Bits abgelegt
    \end{enumerate}
\teil{b}

\aufgabe{2}
\teil{a}
  Eine Entität kann vieles sein. Es beschreibt
\teil{b}
\teil{c}
\teil{d}
\teil{a}
\aufgabe{3}
  

\aufgabe{4}
\aufgabe{5}



\end{document}