\documentclass[11pt,a4paper,DIV=9]{scrartcl}
\usepackage{pgf}
\usepackage{tikz}
\usetikzlibrary{arrows,automata,positioning,shadows}
\usepackage{ngerman}
\usepackage[utf8]{inputenc}
\usepackage{amsmath,amssymb}
\usepackage{tikz-er2}
\usetikzlibrary{shapes,snakes}
% Schriftart ändern
\renewcommand{\rmdefault}{ppl}
%Möglichkeit zur Änderung von Überschriften
\usepackage{sectsty}
%Überschrift \section uandern
\definecolor{blue}{RGB}{76 , 92, 153}
\allsectionsfont{\color{blue}}
\paragraphfont{\color{blue}}

%Variable Blattnummer
\newcommand{\blatt}[1]{
  \newcommand{\blattnr}{#1}
}
%Aufgabe und Aufgabenteil definieren
\newcounter{temp}
\newcommand{\aufgabe}[1]{
  \setcounter{temp}{\value{subsection}}
  \setcounter{subsection}{#1}
  \addtocounter{subsection}{-1}
  \subsection{Aufgabe}
  \setcounter{subsection}{\value{temp}}
}
\newcommand{\teil}[2][]{
  \subsubsection*{#2) #1}
}
\renewcommand{\author}[1]{\renewcommand{\author}{#1}}
\renewcommand{\title}[1]{\renewcommand{\title}{#1}}
\newcommand{\makehomeworktitle}{
  \begin{minipage}[t]{6.5cm}
    \sf{\author}
  \end{minipage}
  \begin{minipage}[t]{6.5cm}
    \begin{flushright}
      \sf{\title\\\today}
    \end{flushright}
  \end{minipage}
  \\[0.2cm]
  \begin{center}
    \sf{
      \color{blue}{
        \LARGE{Aufgabenblatt \blattnr}
      }
    }
  \end{center}
  \vspace{0.1cm}
}

%%%%%%%%%%%%%%%%%%%%%%%%
%%% Statisch
\author{{[}4131658{]} Jan Germann \\{[}1234567{]} Christian Ratz\\Übungsgruppe 1}
\title{Relationale Datenbanken}

%%% Auf jedes Hausaufgabenblatt anpassen
\blatt{3}
%%%%%%%%%%%%%%%%%%%%%%%%
\setcounter{section}{\blattnr}
\begin{document}
\makehomeworktitle

\aufgabe{1}

\aufgabe{2}

\aufgabe{3}

\aufgabe{4}

  Wir haben uns für die Verbindung von \texttt{Villain} zu \texttt{Person} entschiedene, weil wir nicht ausschließen können, dass ein \texttt{Hero} auch (zumindest temporär) ein \texttt{Underling} eines Bösewichts sein kann.\\

  \begin{tikzpicture}[scale=1,every edge/.style={draw=blue, very thick}]
    \node[entity] (person) {Person};
      \node[attribute] () [above left=of person] {\underline{id}}edge node[right]{} (person);
      \node[attribute] () [above =of person] {power-level}edge node[right]{} (person);
      \node[attribute] () [above right=of person] {evilness-level}edge node[right]{} (person);
      \node[attribute] () [right=of person] {name} edge node[right]{} (person);

    \node[relationship] (r3) [left=of person] {hat} edge node[below]{(0,*)} (person);
    \node[isa] (r1) [below=of person] {d} edge node[right]{} (person);
 


    \node[entity] (villain) [below left=of r1] {Villain} edge node[right]{(1,*)} (r3);
    
  
    \node[entity] (hero) [below right=of r1] {Hero};
      \node[relationship] (r2) [right=of hero] {hat} edge node[below]{(0,2)} (hero);
    \node[entity] (sidekick) [above=of r2] {Sidekick} edge node[right]{(1,1)} (r2);




    \node[entity] (home) at ([shift={(-90:4.5)}]hero) {Home};
      \node[attribute] () [below left=of home] {\underline{id}} edge node[right]{} (home);
      \node[attribute] () [left =of home] {power-level} edge node[right]{} (home);
      \node[attribute] () [above left=of home] {evilness-level}edge node[right]{} (home);
      \node[attribute] () [above=of home] {name}edge node[right]{} (home);
    \node[isa] (r4) [below=of home] {o} edge node{} (home);
    \node[entity] (evillair) [below left=of r4] {evil lair};
    \node[entity] (secretbase) [below right=of r4] {secret base};

    \node[relationship] (r5) [below=of evillair] {enthält} edge node[right]{(1,1)} (evillair);
    \node[relationship] (r6) [below=of secretbase] {enthält} edge node[right]{(1,1)} (secretbase);

    \node[entity] (weapon) [below=of r5] {weapon} edge node[right]{(1,1)} (r5);
    \node[entity] (outfit) [below=of r6] {outfit} edge node[right]{(1,1)} (r6);
  \end{tikzpicture}


\end{document}