\documentclass[11pt,a4paper,DIV=9]{scrartcl}
\usepackage{pgf}
\usepackage{tikz}
\usetikzlibrary{arrows,automata,positioning,shadows}
\usepackage{ngerman}
\usepackage[utf8]{inputenc}
\usepackage{amsmath,amssymb}
\usepackage{tikz-er2}
\usetikzlibrary{shapes,snakes}
% Schriftart ändern
\renewcommand{\rmdefault}{ppl}
%Möglichkeit zur Änderung von Überschriften
\usepackage{sectsty}
%Überschrift \section uandern
\definecolor{blue}{RGB}{76 , 92, 153}
\allsectionsfont{\color{blue}}
\paragraphfont{\color{blue}}

%Variable Blattnummer
\newcommand{\blatt}[1]{
  \newcommand{\blattnr}{#1}
}
%Aufgabe und Aufgabenteil definieren
\newcounter{temp}
\newcommand{\aufgabe}[1]{
  \setcounter{temp}{\value{subsection}}
  \setcounter{subsection}{#1}
  \addtocounter{subsection}{-1}
  \subsection{Aufgabe}
  \setcounter{subsection}{\value{temp}}
}
\newcommand{\teil}[2][]{
  \subsubsection*{#2) #1}
}
\renewcommand{\author}[1]{\renewcommand{\author}{#1}}
\renewcommand{\title}[1]{\renewcommand{\title}{#1}}
\newcommand{\makehomeworktitle}{
  \begin{minipage}[t]{6.5cm}
    \sf{\author}
  \end{minipage}
  \begin{minipage}[t]{6.5cm}
    \begin{flushright}
      \sf{\title\\\today}
    \end{flushright}
  \end{minipage}
  \\[0.2cm]
  \begin{center}
    \sf{
      \color{blue}{
        \LARGE{Aufgabenblatt \blattnr}
      }
    }
  \end{center}
  \vspace{0.1cm}
}

%%%%%%%%%%%%%%%%%%%%%%%%
%%% Statisch
\author{{[}4131658{]} Jan Germann \\{[}4054962{]} Christian Ratz\\Übungsgruppe 1}
\title{Relationale Datenbanken}

%%% Auf jedes Hausaufgabenblatt anpassen
\blatt{3}
%%%%%%%%%%%%%%%%%%%%%%%%
\setcounter{section}{\blattnr}
\begin{document}
\makehomeworktitle

\aufgabe{1}
\begin{itemize}
 \item Inheritance
 \item Subclass
 \item Union Type
 \\Manchmal ist es besser, dass die Unterklasse nur von einer einzigen Oberklasse erbt. Dies wird durch ein umkreistes \texttt{u} gekennzeichnet. \\Beispiel: Jedes Auto hat nur einen Besitzer.
\end{itemize}
\aufgabe{2}

\aufgabe{3}
\begin{itemize}
 \item Deleting an entity from a superclass deletes it from all subclasses
 \item A subclass has more attributes than a superclass
 \item In a lattice, there is at least one overlapping specialization
 \item Multi inheritance in an EER-Diagram causes problems
 \item The number of entities in a superclass is equal to the number of entities in its subclasses
 \item A lattice contains at least 4 entity types
\end{itemize}
\aufgabe{4}


\end{document}