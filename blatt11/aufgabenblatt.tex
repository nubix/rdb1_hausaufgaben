\documentclass[11pt,a4paper,DIV=9]{scrartcl}
\usepackage{pgf}
\usepackage{tikz}
\usetikzlibrary{arrows,automata,positioning,shadows}
\usepackage{ngerman}
\usepackage[utf8]{inputenc}
\usepackage{amsmath,amssymb}
\usepackage{tikz-er2}
\usepackage{enumerate}
\usepackage{listings}
\usetikzlibrary{shapes,snakes}
% Schriftart ändern
\renewcommand{\rmdefault}{ppl}
%Möglichkeit zur Änderung von Überschriften
\usepackage{sectsty}
%Überschrift \section uandern
\definecolor{blue}{RGB}{76 , 92, 153}
\allsectionsfont{\color{blue}}
\paragraphfont{\color{blue}}

%Variable Blattnummer
\newcommand{\blatt}[1]{
  \newcommand{\blattnr}{#1}
}
%Aufgabe und Aufgabenteil definieren
\newcounter{temp}
\newcommand{\aufgabe}[1]{
  \setcounter{temp}{\value{subsection}}
  \setcounter{subsection}{#1}
  \addtocounter{subsection}{-1}
  \subsection{Aufgabe}
  \setcounter{subsection}{\value{temp}}
}
\newcommand{\teil}[2][]{
  \subsubsection*{#2) #1}
}
\renewcommand{\author}[1]{\renewcommand{\author}{#1}}
\renewcommand{\title}[1]{\renewcommand{\title}{#1}}
\newcommand{\makehomeworktitle}{
  \begin{minipage}[t]{6.5cm}
    \sf{\author}
  \end{minipage}
  \begin{minipage}[t]{6.5cm}
    \begin{flushright}
      \sf{\title\\\today}
    \end{flushright}
  \end{minipage}
  \\[0.2cm]
  \begin{center}
    \sf{
      \color{blue}{
        \LARGE{Aufgabenblatt \blattnr}
      }
    }
  \end{center}
  \vspace{0.1cm}
}

%%%%%%%%%%%%%%%%%%%%%%%%
%%% Statisch
\author{{[}4131658{]} Jan Germann \\{[}4054962{]} Christian Ratz\\Übungsgruppe 1}
\title{Relationale Datenbanken}

%%% Auf jedes Hausaufgabenblatt anpassen
\blatt{11}
%%%%%%%%%%%%%%%%%%%%%%%%
\setcounter{section}{\blattnr}

\definecolor{dkgreen}{rgb}{0,0.6,0}
\definecolor{gray}{rgb}{0.5,0.5,0.5}
\definecolor{mauve}{rgb}{0.58,0,0.82}

\lstset{ 
  basicstyle=\footnotesize\ttfamily,
  language=sql,                % the language of the code
  numbers=left,
  numberstyle=\tiny\color{gray},
  keywordstyle=\color{blue},          % keyword style
  commentstyle=\color{dkgreen},       % comment style
  stringstyle=\color{mauve}}

\begin{document}
\makehomeworktitle
   \aufgabe{1}
     \begin{enumerate}[a.]
 \item Please create a view called 'movie2000' that contains all movies (id, title) from year 2000.
 \item Given that there is no movie with id 100 contained in the movie table, will the following statements work? Explain your answer.
 \item Is the data contained in the view created in exercise 11.1a physically stored or calculated at query time? How you influence, if the data is stored or calculated?
 \end{enumerate}
\aufgabe{2}
  \begin{enumerate}[a.]
  \item Are the views \textit{females} and \textit{females\_born\_in\_july} as symmetric? Explain your answer.
  \item Given that there is no person with id higher than 4 stored in the persons table. Which of the following tuples can be inserted into the \textit{females\_born\_in\_july} view? Explain your answer.
  \begin{enumerate}[1.] 
  \item (5, 'Bill', 'm', '19.07.1980')
  \item (6, 'Jill', 'w', '21.11.1985')
  \item (7, 'Ann', 'w', '02.07.1979')
  \end{enumerate}
  \end{enumerate}
  \aufgabe{3}
    \begin{enumerate}[a.]
    \item Create. an index on the name column in the person table.
    \item Where is an index more useful? Explain your answer
    \begin{enumerate}[1.]
    \item in a table with high write ratio and low read ratio
    \item in a table with read ratio and low write ratio
    \end{enumerate}
    \item Does it make sense to create an index on the primary key columns of a table? Explain your answer.
    \end{enumerate}
    \aufgabe{4}
      \begin{enumerate}[a.]
      \item How do you suppress the behavior that every statement is executed separately in the first place?
      \item During the transaction you discover an exceptional state and you want to discard all operations you have done until now. How can you do that?
      \item If all operations have been executed successfully, how can you express that you want to save all changes persistently and close the transaction afterrwards? 
      \end{enumerate}
\end{document}

