\documentclass[11pt,a4paper,DIV=9]{scrartcl}
\usepackage{pgf}
\usepackage{tikz}
\usetikzlibrary{arrows,automata,positioning,shadows}
\usepackage{ngerman}
\usepackage[utf8]{inputenc}
\usepackage{amsmath,amssymb}
\usepackage{tikz-er2}
\usepackage{enumerate}
\usepackage{listings}
\usetikzlibrary{shapes,snakes}
% Schriftart ändern
\renewcommand{\rmdefault}{ppl}
%Möglichkeit zur Änderung von Überschriften
\usepackage{sectsty}
%Überschrift \section uandern
\definecolor{blue}{RGB}{76 , 92, 153}
\allsectionsfont{\color{blue}}
\paragraphfont{\color{blue}}

%Variable Blattnummer
\newcommand{\blatt}[1]{
  \newcommand{\blattnr}{#1}
}
%Aufgabe und Aufgabenteil definieren
\newcounter{temp}
\newcommand{\aufgabe}[1]{
  \setcounter{temp}{\value{subsection}}
  \setcounter{subsection}{#1}
  \addtocounter{subsection}{-1}
  \subsection{Aufgabe}
  \setcounter{subsection}{\value{temp}}
}
\newcommand{\teil}[2][]{
  \subsubsection*{#2) #1}
}
\renewcommand{\author}[1]{\renewcommand{\author}{#1}}
\renewcommand{\title}[1]{\renewcommand{\title}{#1}}
\newcommand{\makehomeworktitle}{
  \begin{minipage}[t]{6.5cm}
    \sf{\author}
  \end{minipage}
  \begin{minipage}[t]{6.5cm}
    \begin{flushright}
      \sf{\title\\\today}
    \end{flushright}
  \end{minipage}
  \\[0.2cm]
  \begin{center}
    \sf{
      \color{blue}{
        \LARGE{Aufgabenblatt \blattnr}
      }
    }
  \end{center}
  \vspace{0.1cm}
}

%%%%%%%%%%%%%%%%%%%%%%%%
%%% Statisch
\author{{[}4131658{]} Jan Germann \\{[}4054962{]} Christian Ratz\\Übungsgruppe 1}
\title{Relationale Datenbanken}

%%% Auf jedes Hausaufgabenblatt anpassen
\blatt{11}
%%%%%%%%%%%%%%%%%%%%%%%%
\setcounter{section}{\blattnr}

\definecolor{dkgreen}{rgb}{0,0.6,0}
\definecolor{gray}{rgb}{0.5,0.5,0.5}
\definecolor{mauve}{rgb}{0.58,0,0.82}

\lstset{ 
  basicstyle=\footnotesize\ttfamily,
  language=sql,                % the language of the code
  numbers=left,
  numberstyle=\tiny\color{gray},
  keywordstyle=\color{blue},          % keyword style
  commentstyle=\color{dkgreen},       % comment style
  stringstyle=\color{mauve}}

\begin{document}
\makehomeworktitle
   \aufgabe{1}
     \begin{enumerate}[a.]
 \item Please create a view called 'movie2000' that contains all movies (id, title) from year 2000.
 \begin{lstlisting}
 CREATE VIEW movie2000 AS
 SELECT id, title
 FROM movie
 WHERE year = 2000;
 \end{lstlisting}
 \item Given that there is no movie with id 100 contained in the movie table, will the following statements work? Explain your answer.
 \begin{enumerate}[1.]
 \item 
 \begin{lstlisting}
 INSERT INTO movie2000(id, title) VALUES (100, 'new movie')
 \end{lstlisting}
 Das Jahr darf beim 'INSERT'-Statement ebenfalls nicht NULL sein, da jedoch der VIEW movie2000 nur die ID und den Titel eines Films zul\"asst und das Jahr nicht relevant ist, wird nichts eingef\"ugt. 
 \item 
 \begin{lstlisting}
 UPDATE movie2000 SET title='something else'
 WHERE id=5
 \end{lstlisting}
 Nicht nur der Titel des Films mit der \textit{id} 5 im View \textbf{movie2000} wird ver\"andert, sondern auch der Titel des Films in der Originaltabelle \textbf{movie} wird ver\"andert.
 \end{enumerate}
 \item Is the data contained in the view created in exercise 11.1a physically stored or calculated at query time? How you influence, if the data is stored or calculated? \\\\
 Der erstellte View wird jedes mal bei einer neuen Abfrage ebenfalls neu berechnet, nur der Abfragecode wird physikalisch gespeichert, nicht aber das Ergebnis der Abfrage.
 \end{enumerate}
\aufgabe{2}
  \begin{enumerate}[a.]
  \item Are the views \textit{females} and \textit{females\_born\_in\_july} as symmetric? Explain your answer. \\
  Im ersten CREATE-VIEW-Statement wird ein VIEW f\"ur alle Personen erstellt die weiblich sind.
  Im zweiten CREATE-VIEW-Statement wird ein VIEW f\"ur den ersten VIEW erstellt, der alle weiblichen Personen raussucht, die im Juli Geburtstag haben. Nebenbei enth\"alt das erste Statement eine einfache WHERE-Klausel f\"ur den VIEW, das zweite Statement enth\"alt eine erweiterte WHERE-Klausel mit LIKE. Des Weiteren enth\"alt das zweite Statement ein CHECK-OPTION. Eine Sicht hit Pr\"ufoption veranlasst die Verarbeitung aller Zeilen die in der Selektion f\"ur diese Sicht ge\"andert oder eingef\"ugt wurden.  \item Given that there is no person with id higher than 4 stored in the persons table. Which of the following tuples can be inserted into the \textit{females\_born\_in\_july} view? Explain your answer.
  \begin{enumerate}[1.]  %%%%
  \item (5, 'Bill', 'm', '19.07.1980')
  \\\\Wird nicht eingef\"ugt weil Bill nicht weiblich, sondern m\"annlich ist. Da also die Spalte Geschlecht nicht erf\"ullt wurde, findet auch kein 'INSERT' statt.
  \item (6, 'Jill', 'w', '21.11.1985')
  \\\\Wird ebenfalls nicht eingef\"ugt, weil Jill zwar im Gegensatz zu Bill weiblich ist, jedoch hat sie im November und nicht im Juli ihren Geburtstag. Es findet ebenfalls kein 'INSERT' statt.
  \item (7, 'Ann', 'w', '02.07.1979')
  \\\\Ann wird als einzige Person eingef\"ugt, da sie weiblich ist und im Monat Juli Geburtstag hat. Ihre Daten werden ebenfalls in die Tabelle \textbf{person} und in den View \textbf{females} eingef\"ugt.
  \end{enumerate}
  \end{enumerate}
  \aufgabe{3}
    \begin{enumerate}[a.]
    \item Create an index on the name column in the person table.
    \begin{lstlisting}
    CREATE INDEX person_name
    ON person (name)
    \end{lstlisting}
    \item Where is an index more useful? Explain your answer
    \begin{enumerate}[1.]
    \item in a table with high write ratio and low read ratio
    \item in a table with high read ratio and low write ratio
    \\\\Da ein Index dazu verwendet wird die Datenbankabfrage bzw. den Datenbankabruf schneller zu machen, ist es logischerweise sinnvoller einen Index in einer Tabellen zu erstellen, die nicht so ein schnellen Datenabruf besitzt. Also Tabelle (1).
    \end{enumerate}
    \item Does it make sense to create an index on the primary key columns of a table? Explain your answer. \\\\
    Prim\"arschl\"ussel haben standardm\"a{\ss}ig schon einen Index. Also braucht man keinen Index mehr auf den Prim\"arschl\"ussel.
    \end{enumerate}
    \aufgabe{4}
      \begin{enumerate}[a.]
      \item How do you suppress the behavior that every statement is executed separately in the first place? \\\\
      Durch Speicherpunkte (Savepoints) kann man eine Transaktion in mehrere Teile unterteilen, um Schritt f\"ur Schritt die Datenbank zu ver\"andern. \\\\
      Beispiel:
      \begin{lstlisting}
      DELETE FROM held
      WHERE id = 1;
      
      SAVEPOINT deleted1;
      
      DELETE FROM held
      WHERE id = 2;
      
      ROLLBACK TO deleted1;
      COMMIT;
      \end{lstlisting}
      \item During the transaction you discover an exceptional state and you want to discard all operations you have done until now. How can you do that? \\\\
      Mit einem sogenannten ROLLBACK-Statement (oder auch ROLLBACK WORK). Dadurch wird die aktuelle Transaktion abgebrochen, und die Datenbank wird in den Zustand zur\"uckgesetzt, in dem sie sich vor der Transaktion befand.     
      \item If all operations have been executed successfully, how can you express that you want to save all changes persistently and close the transaction afterwards? \\\\
      Mit einem einfachen COMMIT-Statement (oder auch COMMIT WORK) wird die Transaktion mit Erfolg beendet, alle ge\"anderten Daten in der Datenbank werden gespeichert. \\\\
      Beispiel:
      \begin{lstlisting}
      UPDATE held
      	SET name = 'Karl Heinz-Mueller'
	WHERE name = 'Karl Mueller'
      COMMIT;
      \end{lstlisting}
      \end{enumerate}
\end{document}

